\documentclass{article}  
\usepackage{CJKutf8}
\usepackage{minted}
\begin{document} 
\hfuzz=\maxdimen
\tolerance=10000
\hbadness=10000
\begin{CJK}{UTF8}{gbsn}  
\title{第五章 条件和循环}
\author{}
\date{}
\maketitle
\part*{一、条件语句}
\section*{1.Python之if语句}
\subparagraph*{}
计算机之所以能做很多自动化的任务,因为它可以自己做条件判断。
\subparagraph*{}
输入用户年龄,根据年龄打印不同的年龄,在Python中用if语句实现
\begin{minted}{python}
   age = 20
if age >= 18:
    print 'your age is', age
    print 'adult'
print 'END'
\end{minted}
\subparagraph*{}
注意:Python代码的缩进规则:具有相同缩进的代码被视为代码块,上面的3,4行 print 语句就构成一个代码块(但不包括第5行的print)。如果 if 语句判断为 True,就会执行这个代码块。
\subparagraph*{}
缩进严格按照Python的习惯:4个空格,不要使用Tab,更不要使用混合Tab和空格,否则会因为缩进问题引起的语法错误
\subparagraph*{}
注意:if语句后接表达式,然后用:表示代码块开始
\subparagraph*{}
如果在Python的交互环境下敲代码,还要特别留意缩进,并且退出缩进要多敲一行回车:
\begin{minted}{python}
   age = 20
if age >= 18:
   print 'your age is', age
   print 'adult'
   
   #your age is 20
   #adult
\end{minted}
\section*{2.Python之if-else语句}
\subparagraph*{}
if...else语句:
\begin{minted}{python}
if age >= 18:
    print 'adult'
else:
    print 'teenager'
\end{minted}
\subparagraph*{}
利用 if ... else ... 语句,我们可以根据条件表达式的值为 True 或者 False ,分别执行 if 代码块或者 else 代码块。
\subparagraph*{}
注意:else后面有个“:”
\section*{3.Python之if-elif-else语句}
\subparagraph*{}
一个if...else不够用:需要嵌套if...else语句,但是缩进会很难看
\subparagraph*{}
所以引入if...elif...else语句:
\begin{minted}{python}
if age >= 18:
    print 'adult'
elif age >= 6:
    print 'teenager'
elif age >= 3:
    print 'kid'
else:
    print 'baby'
\end{minted}
\subparagraph*{}
elif就是else if
\subparagraph*{}
特别注意:这一系列的条件判断会从上到下依次判断,如果某个判断为true,执行完对应的代码块,后面的条件判断就直接忽略,不再执行了
\part*{二、循环语句}
\section*{1.Python之for循环}
\subparagraph*{}
list或tuple可以表示一个有序集合。如果我们想依次访问一个list中的每一个元素呢?比如 list:但是元素多了就比较麻烦并且不切实际
\subparagraph*{}
Python的for循环就可以依次把list或Tuple的每个元素迭代出来
\begin{minted}{python}
L = ['Adam', 'Lisa', 'Bart']
for name in L:
    print name
\end{minted}
\subparagraph*{}
注意:name这个变量是在for循环中定义的,意思是,依次取出list的每一个元素,并把元素赋值给name,然后执行for循环(就是缩进的代码块)
\section*{2.Python之while循环}
\subparagraph*{}
while循环不会迭代list或Tuple中的元素,而是根据表达式判断循环是否结束
\begin{minted}{python}
N = 10
x = 0
while x < N:
    print x
    x = x + 1
\end{minted}
\subparagraph*{}
while循环每次先判断x<N,如果为true ,则执行循环的代码块,否则,退出循环
\subparagraph*{}
在循环体内,x=x+1会让x不断增加,最终因为x<N不成立而退出循环
\subparagraph*{}
如果没有这一个语句,while循环在判断 x < N 时总是为True,就会无限循环下去,变成死循环,所以要特别留意while循环的退出条件。
\section*{3.Python之break退出循环}
\subparagraph*{}
用 for 循环或者 while 循环时,如果要在循环体内直接退出循环,可以使用 break 语句。
\subparagraph*{}
计算1至100的整数和:
\begin{minted}{python}
sum = 0
x = 1
while True:
    sum = sum + x
    x = x + 1
    if x > 100:
        break
print sum
\end{minted}
\subparagraph*{}
注意:while True 就是一个死循环,但是在循环体内,我们还判断了 x > 100 条件成立时,用break语句退出循环,这样也可以实现循环的结束。
\subparagraph*{}
练习:利用 while True 无限循环配合 break 语句,计算 1 + 2 + 4 + 8 + 16 + ... 的前20项的和。
\begin{minted}{python}
sum = 0
x = 1
n = 1
while True:
    sum=sum+x
    x=x*2
    n=n+1
    if n>20:
       break
print sum
\end{minted}
\section*{4.Python之continue继续循环}
\subparagraph*{}
continue跳过后续循环代码,继续下一次循环
\subparagraph*{}
利用for循环计算平均分的代码
\begin{minted}{python}
L = [75, 98, 59, 81, 66, 43, 69, 85]
sum = 0.0
n = 0
for x in L:
    sum = sum + x
    n = n + 1
print sum / n
\end{minted}
\subparagraph*{}
只想统计及格分数的平均分,就要把 x < 60 的分数剔除掉,这时,利用 continue,可以做到当 x < 60的时候,不继续执行循环体的后续代码,直接进入下一次循环:
\begin{minted}{python}
for x in L:
    if x < 60:
        continue
    sum = sum + x
    n = n + 1
\end{minted}
\section*{5.Python之多重循环}
\subparagraph*{}
在循环内部,还可以嵌套循环,例如:
\begin{minted}{python}
for x in ['A', 'B', 'C']:
    for y in ['1', '2', '3']:
        print x + y
\end{minted}
\subparagraph*{}
练习:对100以内的两位数,请使用一个两重循环打印出所有十位数字比个位数字小的数,例如,23(2$<$3)
\begin{minted}{python}
for x in [1,2,3,4,5,6,7,8,9]:
    for y in [1,2,3,4,5,6,7,8,9]:
        if x<y:
            print x*10+y
\end{minted}
\end{CJK}
\end{document}