\documentclass{article}  
\usepackage{CJKutf8}
\usepackage{minted}
\begin{document} 
\hfuzz=\maxdimen
\tolerance=10000
\hbadness=10000
\begin{CJK}{UTF8}{gbsn}  
\title{Python3 中文手册}
\author{}
\date{}
\maketitle
\part*{一、开胃菜}
\subparagraph*{}
1.Python 易于使用,是一门完整的编程语言;与 Shell 脚本或批处理文件相比,它为编写大型程序提供了更多的结构和支持。另一方面,Python 提供了比 C 更多的错误检查,并且作为一门 高级语言,它内置支持高级的数据结构类型,例如:灵活的数组和字典。因其更多的通用数据类型,Python 比 Awk 甚至 Perl 都适用于更多问题领域,至少大多数事情在 Python 中与其他语言同样简单。
\subparagraph*{}
2.Python 允许你将程序分割为不同的模块,以便在其他的 Python 程序中重用。Python 内置提供了大量的标准模块,你可以将其用作程序的基础,或者作为学习 Python 编程的示例。这些模块提供了诸如文件 I/O、系统调用、Socket 支持,甚至类似 Tk 的用户图形界面(GUI)工具包接口。
\subparagraph*{}
3.Python 是一门解释型语言,因为无需编译和链接,你可以在程序开发中节省宝贵的时间。Python 解释器可以交互的使用,这使得试验语言的特性、编写临时程序或在自底向上的程序开发中测试方法非常容易。你甚至还可以把它当做一个桌面计算器。
\subparagraph*{}
4.Python 让程序编写的紧凑和可读。用 Python 编写的程序通常比同样的 C、C++ 或 Java 程序更短小,这是因为以下几个原因:
\subparagraph*{}
(1)高级数据结构使你可以在一条语句中表达复杂的操作;
\subparagraph*{}
(2)语句组使用缩进代替开始和结束大括号来组织;
\subparagraph*{}
(3)变量或参数无需声明。
\subparagraph*{}
5.Python 是 可扩展 的:如果你会 C 语言编程便可以轻易地为解释器添加内置函数或模块,或者为了对性能瓶颈作优化,或者将 Python 程序与只有二进制形式的库(比如某个专业的商业图形库)连接起来。一旦你真正掌握了它,你可以将 Python 解释器集成进某个 C 应用程序,并把它当作那个程序的扩展或命令行语言。
\part*{二、使用Python解释器}
\section*{1.调用Python解释器}
\subparagraph*{}
启动Python解释器:
\subparagraph*{}
(1)Python 解释器通常被安装在目标机器的 /usr/local/bin/python3.5 目录下。将 /usr/local/bin 目录包含进 Unix shell 的搜索路径里,以确保可以通过输入:python3.5启动Python解释器
\subparagraph*{}
Unix系统:Control+D结束,让解释器以0码退出
\subparagraph*{}
Python 解释器具有简单的行编辑功能。在 Unix 系统上,任何 Python 解释器都可能已经添加了 GNU readline 库支持,这样就具备了精巧的交互编辑和历史记录等功能。在 Python 主窗口中输入 Control-P 可能是检查是否支持命令行编辑的最简单的方法。
\subparagraph*{}
Python 解释器有些操作类似 Unix shell:当使用终端设备(tty)作为标准输入调用时,它交互的解释并执行命令;当使用文件名参数或以文件作为标准输入调用时,它读取文件并将文件作为脚本执行。
\subparagraph*{}
(2)命令行执行pyhton -c command [arg] ...
\subparagraph*{}
一般建议命令要用单引号包裹起来
\subparagraph*{}
有一些 Python 模块也可以当作脚本使用。你可以使用 python -m module [arg] ... 命令调用它们,这类似在命令行中键入完整的路径名执行 模块 源文件一样。
\subparagraph*{}
使用脚本文件时,经常会运行脚本然后进入交互模式。这也可以通过在脚本之前加上 -i 参数来实现。
\subsection*{(1)参数传递}
\subparagraph*{}
调用解释器时,脚本名和附加参数传入一个名为 sys.argv 的字符串列表。你能够获取这个列表通过执行 import sys,列表的长度大于等于1;没有给定脚本和参数时,它至少也有一个元素:
\subparagraph*{}
1)sys.argv[0] 此时为空字符串);
\subparagraph*{}
2)脚本名指定为 '-' (表示标准输入)时, sys.argv[0] 被设定为 '-');
\subparagraph*{}
3)使用 -c 指令 时,sys.argv[0] 被设定为 '-c');
\subparagraph*{}
4)使用 -m 模块 参数时,sys.argv[0] 被设定为指定模块的全名)
\subparagraph*{}
-c 指令 或者 -m 模块 之后的参数不会被 Python 解释器的选项处理机制所截获,而是留在 sys.argv 中,供脚本命令操作。
\subsection*{(2)交互模式}
\subparagraph*{}
解释器工作于交互模式:从tty读取命令时,根据主提示符来执行,主提示符:($>>>$);继续的部分为从属提示符(...)。
\subparagraph*{}
输入多行语句用从属提示符,例如下面的if语句:
\begin{minted}{python}
    the world is flat=1
    if the world is flat:
        print("Be careful not to fall off!")
\end{minted}
\section*{2.解释器及其环境}
\subsection*{(1)源程序编码}
\subparagraph*{}
默认情况下,Python源文件是UTF-8编码
\subparagraph*{}
在首行后插入至少一行特殊的注释行来定义源文件的编码
\begin{minted}{python}
    # -*- coding:encoding -*-
\end{minted}
\part*{三、Python简介}
\subparagraph*{}
\#\ 表示注释,但是注释不能出现在字符串中
\subparagraph*{}
注意:在练习中遇到的从属提示符表示你需要在最后多输入一个空行,解释器才能知道这是一个多行命令的结束
\begin{minted}{python}
    # this is the first comment
spam = 1  # and this is the second comment
          # ... and now a third!
text = "# This is not a comment because it's inside quotes."
\end{minted}
\section*{1.将Python当做计算器}
\subsection*{(1)数字}
1)int:整数,如2,4,6,10;
\begin{minted}{python}
>>> 2+2
4
>>> 50-5*6
20
>>> 8/5
1.6
\end{minted}
2)float:小数,如5.0,1.6,2.7等\\
3)除法“/”永远返回一个浮点数\\
4)如果使用floor除法并且得到整数结果(丢掉任何小数部分),可以使用//运算符;\\
5)余数:\%{}运算符
\begin{minted}{python}
>>> 17/3
5.666666666666667
>>> 17//3
5
>>> 17%3
2
>>> 5*3+2
17
\end{minted}
6)乘方:**
\begin{minted}{python}
>>> 5**2
25
>>> 2**7
128
\end{minted}
7)变量赋值:=
\begin{minted}{python}
>>> width=20
>>> height=5*9
>>> width*height
900
\end{minted}
变量在使用前必须“定义(赋值)”,否则会出错
\begin{minted}{python}
>>> # try to access an undefined variable
... n
Traceback (most recent call last):
  File "<stdin>", line 2, in <module>
NameError: name 'n' is not defined
\end{minted}
8)浮点数有完整的支持,整数和浮点数的混合计算中,整数会被转换为浮点数
\begin{minted}{python}
>>> 3*3.75/1.5
7.5
>>> 7.0/2
3.5
\end{minted}
9)交互模式下,最近一个表达式的赋值给变量\_{}。这样我们可以把它当作一个桌面计算器,很方便的用于连续计算,例如
\begin{minted}{python}
>>> tax=12.5/100
>>> price=100.50
>>> price*tax
12.5625
>>> price+_
113.0625
>>> round(_,2)
113.06
\end{minted}
此变量对于用户是只读的,不要尝试给它一一赋值,你只会创建一个独立的同名局部变量,它屏蔽了系统内置变量的魔术效果。
10)Python支持其他数字类型,例如Decimal和Fraction,Python还内建了支持复数,使用后缀j或者J表示虚数部分(例如:3+5j)
\subsection*{(2)字符串}
1)字符串用'...'或者"..."来标识\\
“$\backslash$”可以用来做转义字符
\begin{minted}{python}
>>> 'spam eggs'  # single quotes
'spam eggs'
>>> 'doesn\'t'  # use \' to escape the single quote...
"doesn't"
>>> "doesn't"  # ...or use double quotes instead
"doesn't"
>>> '"Yes," he said.'
'"Yes," he said.'
>>> "\"Yes,\" he said."
'"Yes," he said.'
>>> '"Isn\'t," she said.'
'"Isn\'t," she said.'
\end{minted}
2)print()函数可以生成可读性更好的输出,它会省去引号并且打印出转义字符后的特殊字符
\begin{minted}{python}
>>> '"Isn\'t," she said.'
'"Isn\'t," she said.'
>>> print('"Isn\'t," she said.')
"Isn't," she said.
>>> s = 'First line.\nSecond line.'  # \n means newline
>>> s  # without print(), \n is included in the output
'First line.\nSecond line.'
>>> print(s)  # with print(), \n produces a new line
First line.
Second line.
\end{minted}
3)如果前面带有$\backslash$的字符被当做特殊字符,可以用原始字符串,在第一个引号的前面加上r:
\begin{minted}{python}
>>> print('C:\some\name')   # here \n means newline!换行了
C:\some
ame
>>> print(r'C:\some\name')  # note the r before the quote
C:\some\name
\end{minted}
4)多行字符串:"""...""" 或者'''...'''\\
行尾换行符会自动被包含到字符串中,但是可以在行尾加上$\backslash$来避免这个行为
\begin{minted}{python}
print("""\
Usage: thingy [OPTIONS]
     -h                        
     -H hostname                
""")
\end{minted}
5)字符串可以由+操作符连接,可以由*表示重复
\begin{minted}{python}
>>> # 3 times 'un', followed by 'ium'
>>> 3 * 'un' + 'ium'
'unununium'
\end{minted}
相邻的两个字符串文本自动连接在一起
\begin{minted}{python}
>>> 'Py' 'thon'
'Python'
\end{minted}
但是这只用于两个字符串文本,不能用于字符串表达式:
\begin{minted}{python}
>>> prefix = 'Py'
>>> prefix 'thon'  # can't concatenate a variable and a string literal
  ...
SyntaxError: invalid syntax
>>> ('un' * 3) 'ium'
  ...
SyntaxError: invalid syntax
\end{minted}
6)连接多个变量或者连接一个变量和一个字符串文本,使用 +:
\begin{minted}{python}
>>> prefix = 'Py'
>>> prefix + 'thon'
'Python'
\end{minted}
在很想切分很长的字符串时很有用:
\begin{minted}{python}
>>> text = ('Put several strings within parentheses '
            'to have them joined together.')
>>> text
'Put several strings within parentheses to have them joined together.'
\end{minted}
7)字符串也可以被截取(检索)。类似于 C ,字符串的第一个字符索引为 0 。Python没有单独的字符类型;一个字符就是一个简单的长度为1的字符串。:
\begin{minted}{python}
>>> word = 'Python'
>>> word[0]  # character in position 0
'P'
>>> word[5]  # character in position 5
'n'
\end{minted}
索引也可以是负数,这将导致从右边开始计算。例如:
\begin{minted}{python}
>>> word[-1]  # last character
'n'
>>> word[-2]  # second-last character
'o'
>>> word[-6]
'P'
\end{minted}
注意:-0实际上就是0,所以它不会导致从右边开始计算
8)支持切片:索引用于获得单个字符,切片 让你获得一个子字符串:
\begin{minted}{python}
>>> word[0:2]  # characters from position 0 (included) to 2 (excluded)
'Py'
>>> word[2:5]  # characters from position 2 (included) to 5 (excluded)
'tho'
\end{minted}
注意:包含起始的字符,不包含末尾的字符。这使得:
\begin{minted}{python}
>>> word[:2] + word[2:]
'Python'
>>> word[:4] + word[4:]
'Python'
\end{minted}
切片的索引有非常有用的默认值;省略的第一个索引默认为零,省略的第二个索引默认为切片的字符串的大小。:
\begin{minted}{python}
>>> word[:2]  # character from the beginning to position 2 (excluded)
'Py'
>>> word[4:]  # characters from position 4 (included) to the end
'on'
>>> word[-2:] # characters from the second-last (included) to the end
'on'
\end{minted}
记住切片的工作方式:切片时的索引是在两个字符 之间 。左边第一个字符的索引为 0,而长度为 n 的字符串其最后一个字符的右界索引为 n。例如:
\begin{minted}{python}
 +---+---+---+---+---+---+
 | P | y | t | h | o | n |
 +---+---+---+---+---+---+
 0   1   2   3   4   5   6
-6  -5  -4  -3  -2  -1
\end{minted}
文本中的第一行数字给出字符串中的索引点 0…6。第二行给出相应的负索引。切片是从 i 到 j 两个数值标示的边界之间的所有字符。
\subparagraph*{}
对于非负索引,如果上下都在边界内,切片长度就是两个索引之差。例如,word[1:3] 是 2 。
\subparagraph*{}
试图使用太大的索引会导致错误:
\begin{minted}{python}
 >>> word[42]  # the word only has 6 characters
Traceback (most recent call last):
  File "<stdin>", line 1, in <module>
IndexError: string index out of range
\end{minted}
9)Python可以优雅的处理那些没有意义的切片索引:一个过大的索引值(即下标值大于字符串实际长度)将被字符串实际长度所代替,当上边界比下边界大时(即切片左值大于右值)就返回空字符串:
\begin{minted}{python}
>>> word[4:42]
'on'
>>> word[42:]
''
\end{minted}
10)Python的字符串不可以被更改,它们是不可变的,因此,赋值给字符串索引的位置会导致错误:
\begin{minted}{python}
>>> word[0] = 'J'
  ...
TypeError: 'str' object does not support item assignment
>>> word[2:] = 'py'
  ...
TypeError: 'str' object does not support item assignment
\end{minted}
如果需要一个不同的字符串,应该建一个新的:
\begin{minted}{python}
>>> 'J' + word[1:]
'Jython'
>>> word[:2] + 'py'
'Pypy'
\end{minted}
11)内置函数len()返回字符串长度
\begin{minted}{python}
>>> s = 'supercalifragilisticexpialidocious'
>>> len(s)
34
\end{minted}
\subsection*{(3)列表}
1)Python有几个复合数据类型,用来表示其他的值,最常用的是list列表:中括号表示,列表的元素不必是同一类型:
\begin{minted}{python}
>>> squares = [1, 4, 9, 16, 25]
>>> squares
[1, 4, 9, 16, 25]
\end{minted}
2)就像字符串(以及其他所有内建的序列类型一样),列表可以被索引和切片:
\begin{minted}{python}
>>> squares = [1, 4, 9, 16, 25]
>>> squares
[1, 4, 9, 16, 25]
>>> squares[0]    # indexing returns the item
1
>>> squares[-1]
25
>>> squares[-3:]  # slicing returns a new list
[9, 16, 25]
\end{minted}
3)所有的切片操作都会返回一个包含请求元素的新列表,这意味着下面的切片操作会返回一个新的(浅)拷贝副本:
\begin{minted}{python}
>>> squares = [1, 4, 9, 16, 25]
>>> squares
[1, 4, 9, 16, 25]
>>> squares[0]    # indexing returns the item
1
>>> squares[-1]
25
>>> squares[-3:]  # slicing returns a new list
[9, 16, 25]
>>> squares[:]
[1, 4, 9, 16, 25]
\end{minted}
4)列表也支持连接这样的操作:
\begin{minted}{python}
>>> squares = [1, 4, 9, 16, 25]
>>> squares + [36, 49, 64, 81, 100]
[1, 4, 9, 16, 25, 36, 49, 64, 81, 100]
\end{minted}
5)不像不可变的字符串,列表是可变的,它允许修改元素:
\begin{minted}{python}
>>> cubes = [1, 8, 27, 65, 125]  # something's wrong here
>>> 4 ** 3  # the cube of 4 is 64, not 65!
64
>>> cubes[3] = 64  # replace the wrong value
>>> cubes
[1, 8, 27, 64, 125]
\end{minted}
6)使用 append() 方法 (后面我们会看到更多关于列表的方法的内容)在列表的末尾添加新的元素:
\begin{minted}{python}
>>> cubes.append(216)  # add the cube of 6
>>> cubes.append(7 ** 3)  # and the cube of 7
>>> cubes
[1, 8, 27, 64, 125, 216, 343]
\end{minted}
7)也可以对切片赋值,此操作可以改变列表的尺寸,或清空它:
\begin{minted}{python}
>>> letters = ['a', 'b', 'c', 'd', 'e', 'f', 'g']
>>> letters
['a', 'b', 'c', 'd', 'e', 'f', 'g']
>>> # replace some values
>>> letters[2:5] = ['C', 'D', 'E']
>>> letters
['a', 'b', 'C', 'D', 'E', 'f', 'g']
>>> # now remove them
>>> letters[2:5] = []
>>> letters
['a', 'b', 'f', 'g']
>>> # clear the list by replacing all the elements with an empty list
>>> letters[:] = []
>>> letters
[]
\end{minted}
8)内置函数 len() 同样适用于列表:
\begin{minted}{python}
>>> letters = ['a', 'b', 'c', 'd']
>>> len(letters)
4
\end{minted}
9)允许嵌套列表(创建一个包含其它列表的列表),例如:
\begin{minted}{python}
>>> a = ['a', 'b', 'c']
>>> n = [1, 2, 3]
>>> x = [a, n]
>>> x
[['a', 'b', 'c'], [1, 2, 3]]
>>> x[0]
['a', 'b', 'c']
>>> x[0][1]
'b'
\end{minted}
\section*{2.编程的第一步}
写一个生成菲波那契子序列的程序:
\begin{minted}{python}
>>> # Fibonacci series:
... # the sum of two elements defines the next
... a, b = 0, 1
>>> while b < 10:
...     print(b)
...     a, b = b, a+b
...
1
1
2
3
5
8
\end{minted}
这个例子出现的新功能:
\subparagraph*{}
(1)第一行: 多重赋值:变量 a 和 b 同时获得了新的值 0 和 1 ,最后一行又使用了一次。变量赋值前,右边首先完成计算,右边的表达式从左到右计算。
\subparagraph*{}
(2)条件(这里是 b < 10 )为 true 时, while 循环执行。在 Python 中,类似于 C,任何非零整数都是 true;0 是 false。条件也可以是字符串或列表,实际上可以是任何序列;
\subparagraph*{}
(3)所有长度不为零的是 true,空序列是 false。示例中的测试是一个简单的比较。标准比较操作符与 C 相同: $<$ , $>$ , $==$ , $<=$, $>=$ 和 $!=$。
\subparagraph*{}
(4)循环 体 是 缩进 的:缩进是 Python 组织语句的方法。Python (还)不提供集成的行编辑功能,所以你要为每一个缩进行输入 TAB 或空格,一般是4个空格;
\subparagraph*{}
(5)大多数文本编辑器提供自动缩进。交互式录入复合语句时,必须在最后输入一个空行来标识结束(因为解释器没办法猜测你输入的哪一行是最后一行),需要注意的是同一个语句块中的每一行必须缩进同样数量的空白。
\subparagraph*{}
(6)关键字 print() 语句输出给定表达式的值。它控制多个表达式和字符串输出为你想要字符串(就像我们在前面计算器的例子中那样)。
\subparagraph*{}
(7)字符串打印时不用引号包围,每两个子项之间插入空间,所以你可以把格式弄得很漂亮,像这样:
\begin{minted}{python}
>>> i = 256*256
>>> print('The value of i is', i)
The value of i is 65536
\end{minted}
\subparagraph*{}
(8)用一个逗号结尾就可以禁止输出换行:
\begin{minted}{python}
>>> a, b = 0, 1
>>> while b < 1000:
...     print(b, end=',')
...     a, b = b, a+b
...
1,1,2,3,5,8,13,21,34,55,89,144,233,377,610,987,
\end{minted}
\subparagraph*{}
补充:
\subparagraph*{}
(1)因为 ** 的优先级高于 -,所以 -3**2 将解释为 -(3**2) 且结果为 -9。为了避免这点并得到 9,你可以使用 (-3)**2。
\subparagraph*{}
(2)与其它语言不同,特殊字符例如 $\backslash$n 在单引号('...')和双引号("...")中具有相同的含义。两者唯一的区别是在单引号中,你不需要转义 " (但你必须转义 '$\backslash$' ),反之亦然。
\part*{四、深入Python流程控制}
\section*{1.if语句}
\section*{2.for语句}
\section*{3.range()函数}
\section*{4.break和continue语句,以及循环中的else子句}
\section*{5.pass语句}
\section*{6.定义函数}
\section*{7.深入Python函数定义}
\subsection*{(1)默认参数值}
\subsection*{(2)关键字参数}
\subsection*{(3)可变参数列表}
\subsection*{(4)参数列表的分析}
\subsection*{(5)Lambda形式}
\subsection*{(6)文档字符串}
\subsection*{(7)函数注解}
\section*{8.插曲:编码风格}
\end{CJK}
\end{document}