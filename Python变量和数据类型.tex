\documentclass{article}  
\usepackage{CJKutf8}
\usepackage{minted}
\begin{document} 
\hfuzz=\maxdimen
\tolerance=10000
\hbadness=10000
\begin{CJK}{UTF8}{gbsn}  
\title{第三章 Python变量和数据类型}
\author{}
\date{}
\maketitle
\part*{一、Python中的数据类型}
\subparagraph*{}
计算机顾名思义就是可以做数学计算的机器,因此计算机程序理算当然地可以处理各种数值。但是,计算机能处理的不止数值,还可以处理文本,图形,音频,视频等,不同的数据需要定义不同的数据类型。在Python中,能够直接处理的数据类型有以下几种:
\section*{1.整数}
\subparagraph*{}
1)Python可以处理任意大小的整数,也包括负整数;
\subparagraph*{}
2)表示方法:和数学上的写法一模一样,例如:1,,100,-8080,0等等;
\subparagraph*{}
3)计算机由于是二进制,所以,有时候用十六进制表示整数比较方便,十六进制用0x前缀和0~9,a~f表示,例如:0xff00,0xa5b4c3d2等等
\section*{2.浮点数}
\subparagraph*{}
1)浮点数也就是小数,之所以称为浮点数,是因为按照科学记数法表示时,一个浮点数的小数点位置是可以变的,比如,1.23×10\^{}9和12.3×10\^{}8是相等的。
\subparagraph*{}
2)数学写法:1.23,3.14,-9.01等等
\subparagraph*{}
3)对于很大或者很小的浮点数要用科学记数法表示:把10用e代替
\subparagraph*{}
1.23×10\^{}9:1.23e9或者12.3e8;
\subparagraph*{}
0.000012可以写成:1.2e-5
\subparagraph*{}
4)整数和浮点数在计算机内部存储的方式是不同的,整数运算永远是精确到的(包括除法),而浮点数运算则可能是有四舍五入的误差。
\section*{3.字符串}
\subparagraph*{}
1)字符串是以''和""括起来的任意文本,比如'abc',"xyz"等等。
\subparagraph*{}
2)注意:''和""本身只是一种表示方式,不是字符串的一部分,因此,字符串'abc'只有a,b,c这3个字符。
\section*{4.布尔值}
\subparagraph*{}
1)布尔值和布尔代数的表示完全一致,一个布尔值只有True、False两种值,要么是True,要么是False,在Python中,可以直接用True、False表示布尔值(请注意大小写),也可以通过布尔运算计算出来。
\subparagraph*{}
2)布尔值可以用and、or、not运算;
\subparagraph*{}
and运算:与运算,只要所有都为True,and运算结果才是True;
\subparagraph*{}
or运算:或运算,只要其中有一个为True,or运算结果就是True;
\subparagraph*{}
not运算:非运算,它是一个单目运算符,把True变成False,False变成
True。
\section*{5.空值}
\subparagraph*{}
1)空值是Python里一个特殊的值,用None表示;
\subparagraph*{}
2)None不能理解为0,因为0是有意义的,而None是一个特殊的空值。
\subparagraph*{}
3)此外,Python还提供了列表、字典等多种数据类型,还允许创建自定义数据类型,后面会写到。
\subparagraph*{}
练习1:计算十进制整数 45678 和十六进制整数 0x12fd2 之和。
\begin{minted}{python}
    print  45678 + 0x12fd2
\end{minted}
\subparagraph*{}
练习2:请用字符串表示出Learn Python in imooc。
\begin{minted}{python}
    print "Leran Python in imooc"
\end{minted}
\subparagraph*{}
练习3:请计算以下表达式的布尔值(注意==表示判断是否相等):
100$<$99;0xff==255;
\begin{minted}{python}
    print "Leran Python in imooc"
\end{minted}
\part*{二、Python之print语句}
\subparagraph*{}
1.print语句可以向屏幕上输出指定的文字。比如输出'hello,world',用代码实现如下:
\begin{minted}{python}
    print 'hello,world'
\end{minted}
\subparagraph*{}
注意:
\subparagraph*{}
1)当我们在Python交互式环境下编写代码时,>>>是Python解释器的提示符,不是代码的一部分;
\subparagraph*{}
2)当我们在文本编辑器中编写代码时,千万不要自己添加>>>。
\subparagraph*{}
2.print语句也可以跟上多个字符串,用逗号“,”隔开,就可以连成一串输出:
\begin{minted}{python}
    print 'The quick brown fox','jumps over','the lazy dog'
\end{minted}
\subparagraph*{}
3.print语句会依次打印每个字符串,遇到逗号“,”会输出一个空格,因此,输出的字符串是这样的:
\begin{minted}{python}
    The quick brown fox jumps over the lazy dog
\end{minted}
\subparagraph*{}
4.print也可以打印整数,或者计算结果:
\begin{minted}{python}
    print 300
    300  #运行结果
    print 100+200
    300  #运行结果
\end{minted}
\subparagraph*{}
因此,我们可以把计算100+200,Python解释器自动计算出结果300,但是,'100+200='是字符串而非数学公式,Python把它视为字符串,打印如下:
\begin{minted}{python}
    print '100 + 200 =',100+200
    100 + 200 = 300  #运行结果
\end{minted}
\subparagraph*{}
5.练习:请用两种方式打印出 hello, python:
\begin{minted}{python}
    print 'hello, python'
    print 'hello,','python'
    hello, python  #运行结果
\end{minted}
\part*{三、Python的注释}
\subparagraph*{}
任何时候,我们都可以给程序加上注释。注释是用来说明代码的,给自己看也给别人看,而程序运行的时候,Python解释器会直接忽略掉注释,所以,有没有注释不影响程序的执行结果,但是影响到别人能不能看懂你的代码。
\subparagraph*{}
1.Python的注释以\#{}开头,后面的文字直接到行尾都算注释
\begin{minted}{python}
    # 这一行全部都是注释
    print 'hello' #这也是注释
\end{minted}
\subparagraph*{}
2.注释还有一个巧妙的用途,就是一些代码我们不想运行,但是有不想删除,就可以用注释暂时屏蔽掉:
\begin{minted}{python}
    # 暂时不想运行下面一行代码
    # print 'hello,pyhton'
\end{minted}
\part*{四、Python中什么是变量}
\subparagraph*{}
1.在Python中,变量的概念基本上和初中代数的方程变量是一致的。
\subparagraph*{}
2.例如,对于方程式y=x*x,x就是变量。当x=2时,计算结果就是4,当x=5时,计算结果是25。
\subparagraph*{}
3.只是在计算机程序中,变量不仅可以是数字,还可以是任意数据类型。
\subparagraph*{}
4.在Python程序中,变量是用一个变量名表示,变量名必须大小写英文、数字和下划线的组合,且不能以数字开头,比如:
\begin{minted}{python}
    a=1;
    #变量a是一个整数
    t_007='T007'
    #变量t_007是一个字符串
\end{minted}
\subparagraph*{}
5.在Python中,等号=是赋值语句,可以把任意数据类型赋值给变量,同一个变量可以反复赋值,而且可以是不同类型的变量,例如:
\begin{minted}{python}
    a=123;  #a是整数
    print a
    a='imooc' #a变为字符串
    print a
\end{minted}
\subparagraph*{}
这种变量本身类型不固定的语言称之为动态语句,与之对应的是静态语句。
\subparagraph*{}
6.静态语言在定义变量时必须指定变量类型,如果赋值的时候类型不匹配,就会报错。例如:Java是静态语言,赋值语句如下:
\begin{minted}{java}
    int a=123;  //a是整数类型变量
    a="imooc";  //错误:不能把字符串赋给整型变量
\end{minted}
\subparagraph*{}
和静态语言相比,动态语言更灵活,就是这个原因。
\subparagraph*{}
7.请不要把赋值语句的等号等同于数学的等号。比如如下的代码:
\begin{minted}{python}
    x=10
    x=x+2
\end{minted}
\subparagraph*{}
如果从数学上理解x=x+2,是不成立的,在程序中,赋值语句先计算右侧的表达式想x+2,得到12,再赋给变量x。由于x之前的值是10,重新赋值后,x的值变为12。
\subparagraph*{}
8.最后,理解变量在计算机内存中的表示也非常重要。当我们写:a='ABC'时,Python解释器干了两件事情:
\subparagraph*{}
(1)在内存中创建了一个'ABC'的字符串;
\subparagraph*{}
(2)在内存中创建了一个名为a的变量,并把它指向'ABC'。
\subparagraph*{}
9.也可以把一个变量a赋值给另一个变量b,这个操作实际上是把变量b指向变量a所指向的数据,例如下面的代码:
\begin{minted}{python}
     a='ABC'
     b=a
     a='XYZ'
     print b   
\end{minted}
\subparagraph*{}
10.练习:等差数列可以定义为每一项与它的前一项的差等于一个常数,可以用变量 x1 表示等差数列的第一项,用 d 表示公差,请计算数列1 4 7 10 13 16 19 ...前100项的和。
\begin{minted}{python}
     x1=1
     d=3
     n=100
     x100=x1+(n-1)*d    #第100项
     s=(x1+x100)*100/2  #运用前n项和的公式
     print s
\end{minted}
\part*{五、Python中定义字符串}
\subparagraph*{}
前面我们讲了什么是字符串,字符串可以用''或者""括起来表示。
\subparagraph*{}
1.字符串本身包含':可以用""括起来表示
\begin{minted}{python}
   "I'm OK"
\end{minted}
\subparagraph*{}
2.字符串本身包含":可以用''括起来表示
\begin{minted}{python}
   'Learn "Python" in immoc'
\end{minted}
\subparagraph*{}
3.字符串本身既包含'又包含":使用转义字符
\begin{minted}{python}
   'Bob said \"I\'m OK\".'
   #'Bob said "I'm OK".'
   # \ 是一个普通字符,不代表字符串的起始地址
\end{minted}
\subparagraph*{}
注意:转义字符 \ 不计入字符串的内容中。
\subparagraph*{}
常用的转义字符还有:
\begin{minted}{python}
   # \n 表示换行
   # \t 表示一个制表符
   # \\ 表示\字符本身
\end{minted}
\subparagraph*{}
4.练习:请将下面两行内容用Python的字符串表示并打印出来:
Python was started in 1989 by "Guido".Python is free and easy to learn.
\begin{minted}{python}
   s1 = 'Python was started in 1989 by "Guido".'
   print s1
   s2='Python is free and easy to learn.'
   print s2
\end{minted}
\part*{六、Python中raw字符串与多行字符串}
\subparagraph*{}
如果字符串中有很多转义字符,对每一个字符都进行转义比较麻烦,为了避免这种情况,我们可以在字符串前面加个前缀r,表示这是个raw字符串,里面的字符就不需要转义了,例如:
\begin{minted}{python}
    r'\(~_~)/ \(~_~)/'
\end{minted}
\subparagraph*{}
但是r'...'表示法不能表示多行字符串,也不能表示包含'和 "的字符串(为什么?)
\subparagraph*{}
如果要表示多行字符串,可以用'''...'''表示:
\begin{minted}{python}
    '''Line 1
    Line 2
    Line 3'''
\end{minted}
\subparagraph*{}
上面这个字符串的表示方法和下面的是完全一样的:
\begin{minted}{python}
    'Line 1\Line 2\Line 3'
\end{minted}
\subparagraph*{}
还可以在多行字符串前面添加r,把这个多行字符串也变成一个raw字符串:
\begin{minted}{python}
   r'''Python is created by "Guido".
   It is free and easy to learn.
   Let's start learn Python in imooc!''' 
\end{minted}
\subparagraph*{}
练习:请把下面的字符串用r'''...'''的形式改写,并用print打印出来:
\begin{minted}{python}
'\"To be, or not to be\": that is the question.\nWhether it\'s nobler in the mind to suffer.'
\end{minted}
\begin{minted}{python}
   print r'''"To be, or not to be":that is the question.
   Whether it's nobler in the mind to suffer.'''
\end{minted}
\part*{七、Python中的Unicode字符串}
\subparagraph*{}
因为计算机只能处理数字,如果要处理文本,就必须先把文本转换为数字才能处理。最早的计算机在设计时采用8个比特(bit)作为一个字节(byte),所以,一个字节能表示的最大的整数就是255(二进制11111111=十进制255),0 - 255被用来表示大小写英文字母、数字和一些符号,这个编码表被称为ASCII编码,比如大写字母 A 的编码是65,小写字母 z 的编码是122。
\subparagraph*{}
如果要表示中文,显然一个字节是不够的,至少需要两个字节,而且还不能和ASCII编码冲突,所以,中国制定了GB2312编码,用来把中文编进去。
\subparagraph*{}
类似的,日文和韩文等其他语言也有这个问题。为了统一所有文字的编码,Unicode应运而生。Unicode把所有语言都统一到一套编码里,这样就不会再有乱码问题了。
\subparagraph*{}
Unicode通常用两个字节表示一个字符,原有的英文编码从单字节变成双字节,只需要把高字节全部填为0就可以。
\subparagraph*{}
因为Python的诞生比Unicode标准发布的时间还要早,所以最早的Python只支持ASCII编码,普通的字符串'ABC'在Python内部都是ASCII编码的。
\subparagraph*{}
Python在后来添加了对Unicode的支持,以Unicode表示的字符串用u'...'表示,比如:
\begin{minted}{python}
   print u'中文'
   中文
\end{minted}
\subparagraph*{}
注意:不加u,中文就不能正常显示
\subparagraph*{}
Unicode字符串除了多了一个 u 之外,与普通字符串没啥区别,转义字符和多行表示法仍然有效:
\subparagraph*{}
转义:
\begin{minted}{python}
   u'中文\n日文\n韩文'
\end{minted}
\subparagraph*{}
多行:
\begin{minted}{python}
   u'''第一行
   第二行'''
\end{minted}
\subparagraph*{}
raw+多行:
\begin{minted}{python}
   ur'''Python的Unicode字符串支持"中文",
   "日文",
   "韩文"等多种语言'''
\end{minted}
\subparagraph*{}
如果中文字符串在Python环境下遇到 UnicodeDecodeError,这是因为.py文件保存的格式有问题。可以在第一行添加注释
\begin{minted}{python}
   # -*- coding: utf-8 -*-
\end{minted}
\subparagraph*{}
目的是告诉Python解释器,用UTF-8编码读取源代码。然后用Notepad++ 另存为... 并选择UTF-8格式保存。
\subparagraph*{}
练习:用多行Unicode字符串表示下面的唐诗并打印:
\begin{minted}{python}
   # -*- coding: utf-8 -*-
   print '''静夜思
   床前明月光,
   疑似地上霜。
   举头望明月,
   低头思故乡。'''
\end{minted}
\part*{八、Python中整点数和浮点数}
\subparagraph*{}
Python支持对整数和浮点数直接进行四则混合运算,运算规则和数学上的四则运算规则完全一致。
\subparagraph*{}
基本的运算
\begin{minted}{python}
   1 + 2 + 3  #==>6
   4 * 5 - 6  #==>14
   7.5 / 8 + 2.1  #==>3.0375
\end{minted}
\subparagraph*{}
使用括号可以提升优先级,这和数学运算完全一致,注意只能使用小括号,但是括号可以嵌套很多层:
\begin{minted}{python}
   (1 + 2) * 3    # ==> 9
   (2.2 + 3.3) / (1.5 * (9 - 0.3))    #    ==>0.42145593869731807
\end{minted}
\subparagraph*{}
和数学运算不同的地方是,Python的整数运算结果仍然是整数,浮点数运算结果仍然是浮点数:
\begin{minted}{python}
   1 + 2    # ==> 整数 3
   1.0 + 2.0    # ==> 浮点数 3.0
\end{minted}
\subparagraph*{}
但是整数和浮点数混合运算的结果就变成浮点数了:
\begin{minted}{python}
   1 + 2.0    # ==> 浮点数 3.0
\end{minted}
\subparagraph*{}
区分整数运算和浮点数运算的原因:整数运算的结果永远是精确的,而浮点数运算的结果不一定精确,因为计算机内存再大,也无法精确表示出无限循环小数,比如 0.1 换成二进制表示就是无限循环小数。
\subparagraph*{}
整数除不尽的时候:
\begin{minted}{python}
   11 / 4    # ==> 2
\end{minted}
\subparagraph*{}
Python的整数除法,即使除不尽,结果仍然是整数,余数直接被扔掉。不过,Python提供了一个求余的运算 % 可以计算余数:
\begin{minted}{python}
   11 % 4    # ==> 3
\end{minted}
\subparagraph*{}
计算 11 / 4 的精确结果:按照“整数和浮点数混合运算的结果是浮点数”的法则,把两个数中的一个变成浮点数再运算就没问题了。
\begin{minted}{python}
   11.0 / 4    # ==> 2.75
\end{minted}
\subparagraph*{}
练习:请计算 2.5 + 10 / 4 ,并解释计算结果为什么不是期望的 5.0 ?修复上述运算,使得计算结果为5.0
\begin{minted}{python}
   #print 2.5 + 10 / 4
   print 2.5 + 10 / 4.0
\end{minted}
\part*{九、Python中的布尔类型}
\subparagraph*{}
Python支持布尔类型的数据,布尔类型只有True和False两种值;
\subparagraph*{}
布尔类型有以下几种运算:
\subparagraph*{}
(1)与运算and:只有两个布尔值都为True时,计算结果才为True:
\begin{minted}{python}
   True and True   # ==> True
   True and False   # ==> False
   False and True   # ==> False
   False and False   # ==> False
\end{minted}
\subparagraph*{}
(2)或运算or:只要有一个布尔值为True,计算结果就是True:
\begin{minted}{python}
   True or True   # ==> True
   True or False   # ==> True
   False or True   # ==> True
   False or False   # ==> False
\end{minted}
\subparagraph*{}
(3)非运算not:把True变为False,或者把False变为True:
\begin{minted}{python}
   not True   # ==> False
   not False   # ==> True
\end{minted}
\subparagraph*{}
布尔运算在计算机中用来做条件判断,根据计算结果为True或者False,计算机可以自动执行不同的后续代码。
\subparagraph*{}
在Python中,布尔类型还可以与其他数据类型做 and、or和not运算,请看下面的代码:
\begin{minted}{python}
   a = True
   print a and 'a=T' or 'a=F'
\end{minted}
\subparagraph*{}
计算结果不是布尔类型,而是字符串 'a=T',这是为什么呢?
\subparagraph*{}
因为Python把0,空字符串""和None看成False,其他数值和非空字符串都看成True,所以:
\begin{minted}{python}
   True and 'a=T' #计算结果是 'a=T'
   #继续计算 'a=T' or 'a=F' 计算结果还是 'a=T'
\end{minted}
\subparagraph*{}
短路计算:涉及到 and 和 or 运算的一条重要法则
\subparagraph*{}
(1)在计算 a and b 时,如果 a 是 False,则根据与运算法则,整个结果必定为 False,因此返回 a;如果 a 是 True,则整个计算结果必定取决与 b,因此返回 b。
(2) 在计算 a or b 时,如果 a 是 True,则根据或运算法则,整个计算结果必定为 True,因此返回 a;如果 a 是 False,则整个计算结果必定取决于 b,因此返回 b。
\subparagraph*{}
Python解释器在做布尔运算时,只要能提前确定计算结果,它就不会往后算了,直接返回结果。
\subparagraph*{}
练习:
\begin{minted}{python}
   # -*- coding: utf-8 -*-
   a = 'python'
   print 'hello,', a or 'world'
   b = ''
   print 'hello,', b or 'world'
   #因为Python把0、空字符串''和None看成False,其他数值和非空字符串都看成True,而且Python解释器在做布尔运算时,只要能提前确定计算结果,它就不会往后算了,直接返回结果。
   #由于 a = "python" 中 a 不为空值,所以 a or "world" ==> True or True,直接返回第一个 true,即返回 "python"
   #然而 b = ""刚好相反,b为空值,所以 b or "world" ==> False or True, 返回的是第二True。即返回 "world"
\end{minted}
\end{CJK}
\end{document}