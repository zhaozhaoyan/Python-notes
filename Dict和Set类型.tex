\documentclass{article}  
\usepackage{CJKutf8}
\usepackage{minted}
\begin{document} 
\hfuzz=\maxdimen
\tolerance=10000
\hbadness=10000
\begin{CJK}{UTF8}{gbsn}  
\title{第四章 List和Tuple类型}
\author{}
\date{}
\maketitle
\part*{一、dict类型}
\section*{1.什么是dict}
\subparagraph*{}
名字和分数关联起来,形成一个类似的查找表,名字key,成绩value
\begin{minted}{python}
d = {
    'Adam': 95,
    'Lisa': 85,
    'Bart': 59
}
\end{minted}
\subparagraph*{}
花括号{}表示这是一个dict,然后按照key:value写出来即可,最后一个key:value的逗号可以省略
\subparagraph*{}
由于dict也是集合,len()函数可以计算任意集合的大小,一个key:value算一个
\section*{2.访问dict}
\subparagraph*{}
使用d[key]
\subparagraph*{}
list 必须使用索引返回对应的元素,而dict使用key:
\begin{minted}{python}
print d['Adam']
95
print d['Paul']
Traceback (most recent call last):
  File "index.py", line 11, in <module>
   print d['Paul']
KeyError: 'Paul'
\end{minted}
\subparagraph*{}
注意:通过key访问dict的value,只要key存在,dict就返回对应的value,如果key不存在,会直接报错:KeyError
\subparagraph*{}
要避免KeyError发生,有两个办法:
\subparagraph*{}
1)先判断一下key是否存在,用in操作符
\begin{minted}{python}
if 'Paul' in d:
    print d['Paul']
    #如果 'Paul' 不存在,if语句判断为False,自然不会执行 print d['Paul'] ,从而避免了错误。  
\end{minted}
\subparagraph*{}
2)使用dict本身提供的一个get办法,在Key不存在的时候,返回None:
\begin{minted}{python}
print d.get('Bart')
#59
print d.get('Paul')
#None
\end{minted}
\subparagraph*{}
练习:打印出一个dict
\begin{minted}{python}
d = {
    'Adam': 95,
    'Lisa': 85,
    'Bart': 59
}
print 'Adam:', d.get('Adam')
print 'Lisa:', d.get('Lisa')
print 'Bart:', d.get('Bart')
\end{minted}
\section*{3.dict的特点}
\subparagraph*{}
(1)查找速度快,无论dict有10个元素还是10万个元素,查找速度都一样。而list的查找速度随着元素增加而逐渐下降。
\subparagraph*{}
不过dict的查找速度快不是没有代价的,dict的缺点是占用内存大,还会浪费很多内容,list正好相反,占用内存小,但是查找速度慢。
\subparagraph*{}
由于dict是按 key 查找,所以,在一个dict中,key不能重复。
\subparagraph*{}
(2)存储的key-value序对是没有顺序的,和list不一样,
\subparagraph*{}
打印的顺序不一定是我们创建时的顺序,而且,不同的机器打印的顺序都可能不同,这说明dict内部是无序的,不能用dict存储有序的集合。
\subparagraph*{}
(3)作为 key 的元素必须不可变,Python的基本类型如字符串、整数、浮点数都是不可变的,都可以作为 key。但是list是可变的,就不能作为 key。
\subparagraph*{}
不可变这个限制仅作用于key,value是否可变无所谓:
\begin{minted}{python}
{
    '123': [1, 2, 3],  # key 是 str,value是list
    123: '123',  # key 是 int,value 是 str
    ('a', 'b'): True  # key 是 tuple,并且tuple的每个元素都是不可变对象,value是 boolean
}
\end{minted}
\subparagraph*{}
最常用的key还是字符串,因为用起来最方便。
\subparagraph*{}
练习:根据分数索引名字
\begin{minted}{python}
# -*- coding: utf-8 -*-
d = {
    95:'Adam',
    85:'Lisa',
    59:'Bart'
}
print d.get(85)
\end{minted}
\section*{4.更新dict}
\subparagraph*{}
dict是可变的,也就是说,我们可以随时往dict中添加新的 key-value。
\subparagraph*{}
要把新同学'Paul'的成绩 72 加进去,用赋值语句:
\begin{minted}{python}
   d['Paul'] = 72
\end{minted}
\section*{5.遍历dict}
\subparagraph*{}
由于dict也是一个集合,所以,遍历dict和遍历list类似,都可以通过 for 循环实现。
\subparagraph*{}
使用for循环遍历dict中的key:
\begin{minted}{python}
d = { 'Adam': 95, 'Lisa': 85, 'Bart': 59 }
for key in d:
    print key
\end{minted}
\subparagraph*{}
通过key可以获取对应的value的值,因此,在循环体内,可以获取value的值
\subparagraph*{}
练习:使用for循环遍历dict,打印出name:score出来
\begin{minted}{python}
d = {
    'Adam': 95,
    'Lisa': 85,
    'Bart': 59
}
for key in d:
    print key,":",d[key]
\end{minted}
\part*{二、set类型}
\section*{1.什么是set}
\subparagraph*{}
dict的作用是建立一组 key 和一组 value 的映射关系,dict的key是不能重复的。
\subparagraph*{}
当我们只想要 dict 的 key,不关心 key 对应的 value,目的就是保证这个集合的元素不会重复,这时,set就派上用场了。
\subparagraph*{}
set 持有一系列元素,这一点和 list 很像,但是set的元素没有重复,而且是无序的,这点和 dict 的 key很像。
\subparagraph*{}
创建 set 的方式是调用 set() 并传入一个 list,list的元素将作为set的元素:
\begin{minted}{python}
 s = set(['A', 'B', 'C'])
 print s
\end{minted}
\subparagraph*{}
形式类似 list, 但它不是 list,仔细看还可以发现,打印的顺序和原始 list 的顺序有可能是不同的,因为set内部存储的元素是无序的。
\subparagraph*{}
因为set不能包含重复的元素,所以,当我们传入包含重复元素的 list 会怎么样呢?
\begin{minted}{python}
 s = set(['A', 'B', 'C', 'C'])
 print s
 set(['A', 'C', 'B'])
 len(s)
\end{minted}
\subparagraph*{}
结果显示,set会自动去掉重复的元素,原来的list有4个元素,但set只有3个元素。
\section*{2.访问set}
\subparagraph*{}
由于set存储的是无序集合,所以我们没法通过索引来访问。
\subparagraph*{}
访问 set中的某个元素实际上就是判断一个元素是否在set中。
\subparagraph*{}
用 in 操作符判断:注意大小写
\begin{minted}{python}
 s = set(['Adam', 'Lisa', 'Bart', 'Paul'])
 'Bart' in s
  #True
  'Bill' in s
  #False
  'bart' in s
  #False
\end{minted}
\subparagraph*{}
练习:请改进set,使得 'adam' 和 'bart'都能返回True。
\begin{minted}{python}
s = set([name.lower() for name in ['Adam', 'Lisa', 'Bart', 'Paul']])
print 'adam' in s
print 'bart' in s
# L=['Adam', 'Lisa', 'Bart', 'Paul']
# M=[]
# for x in L:
#     y=x.lower()
#     print x
#     M.append(y)
# s = set(M)
# print 'adam' in s
# print 'bart' in s
\end{minted}
\section*{3.set的特点}
\subparagraph*{}
(1)set的内部结构和dict很像,唯一区别是不存储value,因此,判断一个元素是否在set中速度很快。
\subparagraph*{}
(2)set存储的元素和dict的key类似,必须是不变对象,因此,任何可变对象是不能放入set中的
\subparagraph*{}
(3)set存储的元素也是没有顺序的。
\begin{minted}{python}
weekdays = set(['MON', 'TUE', 'WED', 'THU', 'FRI', 'SAT', 'SUN'])
x = '???' # 用户输入的字符串
if x in weekdays:
    print 'input ok'
else:
    print 'input error'
\end{minted}
\section*{4.遍历set}
\subparagraph*{}
for循环
\begin{minted}{python}
s = set(['Adam', 'Lisa', 'Bart'])
for name in s:
    print name
\end{minted}
\subparagraph*{}
注意:for循环在遍历set时,元素的顺序很可能是不同的,而且不同机器上运行的结果也可能会是不同的
\subparagraph*{}
练习: for 循环遍历如下的set,打印出 name: score
\begin{minted}{python}
s = set(['Adam', 'Lisa', 'Bart'])
for name in s:
    print name
\end{minted}
\section*{5.更新set}
\subparagraph*{}
由于set存储的是一组不重复的无序元素,因此,更新set主要做两件事:
\subparagraph*{}
(1)把新的元素添加到set中,(2)把已有元素从set中删除。
\subparagraph*{}
添加元素时,用set的add()方法
\begin{minted}{python}
s = set([1, 2, 3])
s.add(4)
print s
set([1, 2, 3, 4])
\end{minted}
\subparagraph*{}
如果添加的元素已经存在于set中,add()不会报错,但是不会加进去了:
\begin{minted}{python}
s = set([1, 2, 3])
s.add(3)
print s
set([1, 2, 3])
\end{minted}
\subparagraph*{}
删除set中的元素时,用set的remove()方法:
\begin{minted}{python}
s = set([1, 2, 3, 4])
s.remove(4)
print s
#set([1, 2, 3])
\end{minted}
\subparagraph*{}
如果删除的元素不存在set中,remove()会报错:
\begin{minted}{python}
s = set([1, 2, 3])
s.remove(4)
Traceback (most recent call last):
  File "<stdin>", line 1, in <module>
KeyError: 4
\end{minted}
\subparagraph*{}
所以用add()可以直接添加,而remove()前需要判断。
\subparagraph*{}
练习:针对下面的set,给定一个list,对list中的每一个元素,如果在set中,就将其删除,如果不在set中,就添加进去。
s = set(['Adam', 'Lisa', 'Paul'])
L = ['Adam', 'Lisa', 'Bart', 'Paul']
\begin{minted}{python}
s = set(['Adam', 'Lisa', 'Paul'])
L = ['Adam', 'Lisa', 'Bart', 'Paul']
for name in L:
    if name in s:
        s.remove(name)
    else:
        s.add(name)
print s
\end{minted}
\end{CJK}
\end{document}