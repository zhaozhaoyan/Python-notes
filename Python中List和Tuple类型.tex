\documentclass{article}  
\usepackage{CJKutf8}
\usepackage{minted}
\begin{document} 
\hfuzz=\maxdimen
\tolerance=10000
\hbadness=10000
\begin{CJK}{UTF8}{gbsn}  
\title{第四章 List和Tuple类型}
\author{}
\date{}
\maketitle
\part*{一、List类型}
\section*{1.Python创建List}
\subparagraph*{}
Python内置的一种数据类型是列表:list;
\subparagraph*{}
list是一种有序的集合,可以随时添加和删除其中的元素。
\subparagraph*{}
比如,列出班里所有同学的名字,就可以用一个list表示:
\begin{minted}{python}
   L=['Michael', 'Bob', 'Tracy']
   print L
\end{minted}
\subparagraph*{}
list是数学意义上的有序集合,也就是说,list中的元素是按照顺序排列的。
\subparagraph*{}
list的构造:直接用 [ ] 把list的所有元素都括起来,就是一个list对象。通常,我们会把list赋值给一个变量,这样,就可以通过变量来引用list:
\begin{minted}{python}
   classmates = ['Michael', 'Bob', 'Tracy']
   print classmates # 打印classmates变量的内容
   #输出结果:['Michael', 'Bob', 'Tracy']
\end{minted}
\subparagraph*{}
由于Python是动态语言,所以list中包含的元素并不要求都必须是同一种数据类型,我们完全可以在list中包含各种数据:
\begin{minted}{python}
   L = ['Michael', 100, True]
   print L
\end{minted}
\subparagraph*{}
一个元素也没有的list,就是空list:
\begin{minted}{python}
   empty_list = []
\end{minted}
\subparagraph*{}
练习:假设班里有3名同学:Adam,Lisa和Bart,他们的成绩分别是 95.5,85 和 59,请按照 名字, 分数, 名字, 分数... 的顺序按照分数从高到低用一个list表示,然后打印出来
\begin{minted}{python}
   L = ['Adma',95.5,'Lisa',85,'Bart',59]
   print L
\end{minted}
\section*{2.Python按照索引访问List}
\subparagraph*{}
由于list是一个有序集合,所以,我们可以用一个list按分数从高到低表示出班里的3个同学:
\begin{minted}{python}
   L = ['Adam', 'Lisa', 'Bart']
   print L
\end{minted}
\subparagraph*{}
那我们如何从list中获取指定第 N 名的同学呢?方法是通过索引来获取list中的指定元素。
\subparagraph*{}
特别注意:索引从 0 开始,也就是说,第一个元素的索引是0,第二个元素的索引是1,以此类推。
\subparagraph*{}
因此,要打印第一名同学的名字,用 L[0]:
\begin{minted}{python}
   L = ['Adam', 'Lisa', 'Bart']
   print L[0]
\end{minted}
\subparagraph*{}
要打印第二名同学的名字,用 L[1]:
\begin{minted}{python}
   L = ['Adam', 'Lisa', 'Bart']
   print L[1]
\end{minted}
\subparagraph*{}
要打印第三名同学的名字,用 L[2]:
\begin{minted}{python}
   L = ['Adam', 'Lisa', 'Bart']
   print L[2]
\end{minted}
\subparagraph*{}
要打印第四名同学的名字,用 L[3]:
\begin{minted}{python}
   L = ['Adam', 'Lisa', 'Bart']
   print L[3]
   Traceback (most recent call last):
   File "<stdin>", line 1, in <module>
   IndexError: list index out of range
\end{minted}
\subparagraph*{}
报错了!IndexError意思就是索引超出了范围,因为上面的list只有3个元素,有效的索引是 0,1,2。
\subparagraph*{}
注意:使用索引时,千万注意不要越界。
\subparagraph*{}
练习:三名同学的成绩可以用一个list表示:
L = [95.5, 85, 59]请按照索引分别打印出第一名、第二名、第三名,同时测试 print L[3]。
\begin{minted}{python}
   L = [95.5,85,59]
   print L[0]
   print L[1]
   print L[2]
\end{minted}
\section*{3.Python之倒序访问List}
\subparagraph*{}
我们还是用一个list按分数从高到低表示出班里的3个同学:
\begin{minted}{python}
   L = ['Adam', 'Lisa', 'Bart']
   print L
\end{minted}
\subparagraph*{}
这时,老师说,请分数最低的同学站出来。
\subparagraph*{}
要写代码完成这个任务,我们可以先数一数这个 list,发现它包含3个元素,因此,最后一个元素的索引是2:
\begin{minted}{python}
   L = ['Adam', 'Lisa', 'Bart']
   print L[2]
\end{minted}
\subparagraph*{}
更简单的方法?
\subparagraph*{}
Bart同学是最后一名,俗称倒数第一,所以,我们可以用 -1 这个索引来表示最后一个元素:
\begin{minted}{python}
   L = ['Adam', 'Lisa', 'Bart']
   print L[-1]
\end{minted}
\subparagraph*{}
类似的,倒数第二用 -2 表示,倒数第三用 -3 表示,倒数第四用 -4 表示:
\begin{minted}{python}
   L = ['Adam', 'Lisa', 'Bart']
   print L[-1]
   print L[-2]
   print L[-3]
   print L[-4]
   Traceback (most recent call last):
   File "<stdin>", line 1, in <module>
   IndexError: list index out of range
\end{minted}
\subparagraph*{}
L[-4] 报错了,因为倒数第四不存在,一共只有3个元素。
\subparagraph*{}
使用倒序索引时,也要注意不要越界。
\subparagraph*{}
练习:三名同学的成绩可以用一个list表示:L = [95.5, 85, 59]
请按照倒序索引分别打印出倒数第一、倒数第二、倒数第三。
\begin{minted}{python}
   L = [95.5, 85, 59]
   print L[-1]
   print L[-2]
   print L[-3]
\end{minted}
\section*{4.Python之添加新元素}
\begin{minted}{python}
   L = ['Adam', 'Lisa', 'Bart']
   print L
\end{minted}
\subparagraph*{}
添加新元素Paul:
\subsection*{(1)append()添加到尾部}
\subparagraph*{}
append()总是把新的元素添加到list的尾部
\begin{minted}{python}
   L = ['Adam', 'Lisa', 'Bart']
   L.append('Pual')
   print L
   #['Adam','Lisa','Bart','Pual']
\end{minted}
\subsection*{(2)insert()添加元素}
\subparagraph*{}
insert(),接受两个参数,第一个参数是索引导,第二个参数是待添加的新元素
\begin{minted}{python}
   L = ['Adam', 'Lisa', 'Bart']
   L.insert(0,'Pual')
   print L
   #['Pual','Adam','Lisa','Bart']
\end{minted}
\subparagraph*{}
L.insert(0,'Pual')的意思是,'Pual'将被添加到索引为0的位置上(也就是第一个),而原来索引为0的Adma同学,都自动向后移一位。
\subparagraph*{}
练习:假设新来的一名学生Pual,Pual同学的成绩比Bart好,但是比Lisa差,他应该排到第三名的位置,请用代码实现。
\begin{minted}{python}
   L = ['Adam', 'Lisa', 'Bart']
   L.insert(2,'Pual')
   print L
  #方法一
   #L.insert(2,'Paul')    
   #print L
  #方法二
   #a=Paul
   #L.insert(a>L[1],a<L[-1})
   #print L
  #方法三
   #L = ['Adam','Lisa','Paul','Bart']
   #print L
  #方法四
   #L.insert(-2,'Paul')
   #print L
\end{minted}
\section*{5.Python之删除元素}
\subparagraph*{}
使用pop()删除元素
\begin{minted}{python}
   L=['Pual','Adam','Lisa','Bart']
   L.pop()
   'Pual'
   print L
   #['Adma','Lisa','Bart']
\end{minted}
\subparagraph*{}
pop()方法总是删掉list的最后一个元素,并且它还返回这个元素,所以我们执行L.pop()后,会打印出'Pual'。
\subparagraph*{}
如果Pual在第三位:
\subparagraph*{}
必须先定位Pual的位置,由于Paul的索引是2,所以用pop(2)删除:
\begin{minted}{python}
   L=['Adam','Lisa','Pual','Bart']
   L.pop(2)
   'Pual'
   print L
   #['Adma','Lisa','Bart']
\end{minted}
\subparagraph*{}
练习:L = ['Adam', 'Lisa', 'Paul', 'Bart']
Paul的索引是2,Bart的索引是3,如果我们要把Paul和Bart都删掉,请解释下面的代码为什么不能正确运行:
L.pop(2)L.pop(3)怎样调整代码可以把Paul和Bart都正确删除掉?
\begin{minted}{python}
   L=['Adam','Lisa','Pual','Bart']
   L.pop(2)
   L.pop(2)
   print L
\end{minted}
\section*{6.Python之替换元素}
\begin{minted}{python}
   L=['Adam','Lisa','Bart']
\end{minted}
\subparagraph*{}
1)直接用Pual把Bart替换掉
\begin{minted}{python}
   L=['Adam','Lisa','Bart']
   L[2]='Pual'
   print L
\end{minted}
\subparagraph*{}
对list中的某一个索引赋值,就可以直接用新的元素替换掉原来的元素,list包含的元素个数保持不变。
\subparagraph*{}
2)Bart还可以用-1做索引,因此,下面的代码也可以完成同样的工作:
\begin{minted}{python}
   L=['Adam','Lisa','Bart']
   L[-1]='Pual'
   print L
\end{minted}
\subparagraph*{}
练习:班里的同学按照分数排名是这样的:L = ['Adam', 'Lisa', 'Bart']但是,在一次考试后,Bart同学意外取得第一,而Adam同学考了倒数第一。请通过对list的索引赋值,生成新的排名。
\begin{minted}{python}
   L=['Adam','Lisa','Bart']
   a=L[0]
   L[0]=L[2]
   L[2]=a
   print L
\end{minted}
\part*{二、Tuple类型}
\section*{1.Python之创建tuple}
\subparagraph*{}
tuple是另一种有序的列表,中文翻译是”元组“。tuple和list非常相似,tuple一旦创建完毕,就不能修改了。
\subparagraph*{}
同样是表示班里的同学,用tuple表示如下:
\begin{minted}{python}
   t=('Adam','Lisa','Bart')
\end{minted}
\subparagraph*{}
创建tuple和创建list唯一不同的是用()代替了[]
\subparagraph*{}
但是,这个t就不能改变了,tuple没有append()方法,也没有insert()和pop()方法,所以,新的元素没法直接往tuple中添加,元素想退出tuple也不行。
\subparagraph*{}
获取tuple元素的方式和list的一模一样,我们可以正常使用t[0],t[-1]等索引方式访问元素,但是不能赋值成别的元素:
\begin{minted}{python}
   t=('Adam','Lisa','Bart')
   t[0]='Pual'
   print t
   Traceback (most recent call last):
   File "<stdin>", line 1, in <module>
   TypeError: 'tuple' object does not support item     assignment
   #程序报错
\end{minted}
\subparagraph*{}
练习:创建一个tuple,顺序包含0-9这10个数
\begin{minted}{python}
   t=(0,1,2,3,4,5,6,7,8,9)
   print t
\end{minted}
\section*{2.Python之创建tuple}
\subparagraph*{}
tuple和list一样,可以包含0个、1个和任意多个元素。
\subparagraph*{}
包含多个元素的 tuple,前面我们已经创建过了。
\subparagraph*{}
空tuple:包含0个元素的tuple,直接用()表示
\begin{minted}{python}
   t=()
   print t
\end{minted}
\subparagraph*{}
创建包含1个元素的 tuple 呢?来试试:
\begin{minted}{python}
   t=(1)
   print t
\end{minted}
\subparagraph*{}
好像哪里不对!t 不是 tuple ,而是整数1。为什么呢?
\subparagraph*{}
因为()既可以表示tuple,又可以作为括号表示运算时的优先级,结果 (1) 被Python解释器计算出结果 1,导致我们得到的不是tuple,而是整数 1。
\subparagraph*{}
正是因为用()定义单元素的tuple有歧义,所以 Python 规定,单元素 tuple要多加一个逗号“,”,这样就避免了歧义:
\begin{minted}{python}
   t=(1,)
   print t
\end{minted}
\subparagraph*{}
Python在打印单元素tuple时,也自动添加了一个“,”,为了更明确地告诉你这是一个tuple。
\subparagraph*{}
多元素 tuple 加不加这个额外的“,”效果是一样的:
\begin{minted}{python}
   t=(1,2,3,)
   print t
   #输出结果(1,2,3)
\end{minted}
\subparagraph*{}
练习:请指出右边编辑器中代码为什么没有创建出包含一个学生的 tuple:t = ('Adam')print t请修改代码,确保 t 是一个tuple。
\begin{minted}{python}
   t=('Adam',)
   print t
   #输出结果('Adam')
\end{minted}
\section*{3.Python之可变的tuple}
\subparagraph*{}
前面我们看到了tuple一旦创建就不能修改。现在,我们来看一个“可变”的tuple:
\begin{minted}{python}
   t=('a','b',['A','B'])
   print t)
\end{minted}
\subparagraph*{}
注意到 t 有 3 个元素:'a','b'和一个list:['A', 'B']。list作为一个整体是tuple的第3个元素。list对象可以通过 t[2] 拿到:
\begin{minted}{python}
   L=t[2]
\end{minted}
\subparagraph*{}
然后,我们把list的两个元素改一改:
\begin{minted}{python}
   L[0]='X'
   L[1]='Y'
\end{minted}
再看看tuple的内容:
\begin{minted}{python}
   print t
   # 输出结果('a','b',['X','Y'])
\end{minted}
\subparagraph*{}
不是说tuple一旦定义后就不可变了吗?怎么现在又变了?
\subparagraph*{}
表面上看,tuple的元素确实变了,但其实变的不是 tuple 的元素,而是list的元素。
\subparagraph*{}
tuple一开始指向的list并没有改成别的list,所以,tuple所谓的“不变”是说,tuple的每个元素,指向永远不变。即指向'a',就不能改成指向'b',指向一个list,就不能改成指向其他对象,但指向的这个list本身是可变的!
\subparagraph*{}
理解了“指向不变”后,要创建一个内容也不变的tuple怎么做?那就必须保证tuple的每一个元素本身也不能变。
\subparagraph*{}
练习:定义了tuple:t = ('a', 'b', ['A', 'B'])由于 t 包含一个list元素,导致tuple的内容是可变的。能否修改上述代码,让tuple内容不可变?
\begin{minted}{python}
   #t=('a','b',['A','B'])
   t=('a','b',('A','B'))
   print t
\end{minted}
\end{CJK}
\end{document}